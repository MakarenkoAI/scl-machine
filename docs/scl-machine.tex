\scnheader{Программный вариант реализации логической машины интерпретации логических sc-моделей компьютерных систем}
\scnidtf{SCL-machine}
\scnidtf{ostis-inference}
\begin{scnrelfromset}{декомпозиция программной системы}
    \scnitem{База знаний SCL-machine}
    \scnitem{Решатель задач SCL-machine}
    \scnitem{Интерфейс SCL-machine}
\end{scnrelfromset}

\begin{scnrelfromset}{реализованные логические связки}
    \scnitem{импликация*}
    \scnitem{дизъюнкция*}
    \scnitem{конъюнкция*}
    \scnitem{отрицание*}
\end{scnrelfromset}

\begin{scnrelfromset}{не реализованные логические связки}
    \scnitem{эквиваленция*}
    \scnitem{строгая дизъюнкция*}
\end{scnrelfromset}

\scnheader{Решатель задач SCL-machine}
\begin{scnrelfromset}{обобщённая декомпозиция}
    \scnitem{Агент прямого логического вывода}
    \scnitem{Агент обратного логического вывода}
    \begin{scnindent}
        \scntext{примечание}{Не реализовано.}
    \end{scnindent}
\end{scnrelfromset}

\scnheader{Агент прямого логического вывода}
\begin{scnindent}
    \scnrelfrom{пример входной конструкции}{\scnfileimage[40em]{images/direct_inference_input.png}}
    \begin{scnrelfromvector}{аргументы агента}
        \scnitem{\_target\_template}
        \begin{scnindent}
            \scnidtf{targetTemplate}
            \scnidtf{targetStatement}
            \scnidtf{ожидаемый результат выполнения логического вывода}
            \scntext{пояснение}{Шаблон, успешный поиск которого показывает, что цель логического вывода достигнута и применение правил можно прекратить.}
            \scnrelfrom{описание примера}{\scnfileimage[28em]{images/target_template.png}}
        \end{scnindent}
        \scnitem{\_rule\_set}
        \begin{scnindent}
            \scnidtf{ruleSet}
            \scntext{пояснение}{Ориентированное множество множеств правил, применяя которые требуется совершить логический вывод.
            Первым элементом множества является множество правил, которые применяются в первую очередь, а каждое следующее множество
            правил применяется после предыдущего. Таким образом указываются приоритеты множеств правил.}
            \scnrelfrom{описание примера}{\scnfileimage[27em]{images/rules_set.png}}
            \begin{scnindent}
                \scnrelfrom{описание примера}{\scnfileimage[37em]{images/rule_example.png}}
            \end{scnindent}
        \end{scnindent}
        \scnitem{\_argument\_set}
        \begin{scnindent}
            \scnidtf{argumentSet}
            \scntext{пояснение}{Множество тех элементов, которые должны быть подставлены как значение переменных шаблона цели.}
            \begin{scnindent}
                \scntext{пояснение}{В данном примере значением переменной \_x в шаблоне цели может быть sc-узел
                argument\_1, тогда значением \_y будет argument\_2. Также значением переменной \_x может быть
                sc-узел argument\_2, а значением \_y -- argument\_1.}
            \end{scnindent}
            \scnrelfrom{описание примера}{\scnfileimage[30em]{images/argument_set.png}}
        \end{scnindent}
    \end{scnrelfromvector}
    \scnrelfrom{пример выходной конструкции}{\scnfileimage[30em]{images/direct_inference_output.png}}
    \begin{scnrelfromvector}{обобщённый алгоритм}
        \scnfileitem{Получение параметров агента, проверка их валидности. Вызов агента;}
        \scnfileitem{Проверка, достигнута ли уже цель в базе знаний;}
        \begin{scnindent}
            \scntext{примечание}{Выполняется поиск по шаблону target template с параметрами шаблона arguments set.}
        \end{scnindent}
        \scnfileitem{Построение вектора очереди правил на основе множества правил. Цикл по всем правилам и пока не достигнута цель;}
        \begin{scnindent}
            \begin{scnrelfromvector}{циклические операции}
                \scnfileitem{Получение посылки логического правила;}
                \scnfileitem{Определение типа посылки (связка конъюнкци, дизъюнкции, отрицания или атомарная логическая формула);}
                \scnfileitem{Проверка истинности посылки в зависимости от её типа;}
                \begin{scnindent}
                    \scntext{замечание}{Конъюнкция, дизъюнкция, отрицание работают нестабильно.}
                \end{scnindent}
                \scnfileitem{Генерация по шаблону следствия;}
                \scnfileitem{Добавление в дерево решений узла правила.}
                \begin{scnindent}
                    \scntext{примечание}{Смотрите пример выходной конструкции.}
                \end{scnindent}
            \end{scnrelfromvector}
        \end{scnindent}
        \scnfileitem{Возврат дерева применённых правил.}
    \end{scnrelfromvector}
\end{scnindent}

\scnheader{Агент прямого логического вывода}
\begin{scnindent}
    \scnrelfrom{пример входной конструкции}{\scnfileimage[20em]{images/direct_inference_input_4.png}}
    \begin{scnrelfromvector}{аргументы агента}
        \scnitem{\_rule\_set}
        \begin{scnindent}
            \scnidtf{ruleSet}
            \scntext{пояснение}{Ориентированное множество множеств правил, применяя которые требуется совершить логический вывод.
            Первым элементом множества является множество правил, которые применяются в первую очередь, а каждое следующее
            множество правил применяется после предыдущего. Таким образом указываются приоритеты множеств правил.}
            \scnrelfrom{описание примера}{\scnfileimage[27em]{images/rules_set.png}}
            \begin{scnindent}
                \scnrelfrom{описание примера}{\scnfileimage[37em]{images/rule_example.png}}
            \end{scnindent}
        \end{scnindent}
        \scnitem{\_input\_structure}
        \begin{scnindent}
            \scnidtf{inputStructure}
            \scntext{пояснение}{Структура, элементы которой используются при применении правил. Каждый sc-узел этой структуры
            добавляется во множество аргументов, которые должны быть подставлены как значение переменных шаблона цели.}
            \begin{scnindent}
                \scnrelfrom{аналог}{\_argument\_set}
                \scnrelfrom{примечание}{можно использовать структуру всей базы знаний системы, например, sc-узел \scnkeyword{База знаний IMS}.}
            \end{scnindent}
            \scnrelfrom{описание примера}{\scnfileimage[27em]{images/input_structure.png}}
        \end{scnindent}
        \scnitem{\_output\_structure}
        \begin{scnindent}
            \scnidtf{outputStructure}
            \scntext{пояснение}{Структура, в которую добавляются сгенерированные в ходе логического вывода конструкции.}
            \begin{scnindent}
                \scnrelfrom{примечание}{можно использовать структуру всей базы знаний системы, например, sc-узел \scnkeyword{База знаний IMS}.}
            \end{scnindent}
            \scnrelfrom{описание примера}{\scnfileimage[27em]{images/output_structure.png}}
        \end{scnindent}
        \scnitem{\_target\_template}
        \begin{scnindent}
            \scnidtf{targetTemplate}
            \scnidtf{targetStatement}
            \scnidtf{ожидаемый результат выполнения логического вывода}
            \scntext{пояснение}{Шаблон, успешный поиск которого показывает, что цель логического вывода достигнута и применение правил можно прекратить.}
            \scnrelfrom{описание примера}{\scnfileimage[28em]{images/target_template.png}}
        \end{scnindent}
    \end{scnrelfromvector}
    \scnrelfrom{пример выходной конструкции}{\scnfileimage[35em]{images/direct_inference_output.png}}
    \scntext{примечание}{Работа агента заключается в последовательном применении правил из входного множества правил,
        генерируя структуры, если атомарная формула принадлежит классу \scnkeyword{concept\_formula\_for\_generation}. Если правило
        применилось успешно, то создаётся отношение \scnkeyword{nrel\_satisfiable\_formula} между правилом и моделью, на которой это
        правило выполнимо. Если правило применилось безуспешно, то оно добавляется во множество безуспешно применённых
        правил, которые применяются повторно в случае успешного применения какого-либо другого правила. Также после
        каждого успешного применения правила проверяется, достигнута ли цель(если цель передалась), и, если цель достигнута,
        выполнение агента завершается успешно и остальные правила не применяются.}
    \begin{scnrelfromvector}{обобщённый алгоритм}
        \scnfileitem{Получение параметров агента, вызов агента;}
        \scnfileitem{Получение всех sc-узлов из inputStructure, если структура валидна, заполнение ими списка аргументов;}
        \scnfileitem{Проверка, достигнута ли уже цель в базе знаний с полученными аргументами;}
        \begin{scnindent}
            \scntext{примечание}{Выполняется поиск по шаблону target template с параметрами структуры \_input\_structure.
            Если шаблон найден, агент завершает работу, возвращает узел, принадлежащий \scnkeyword{concept\_success\_solution}.}
        \end{scnindent}
        \scnfileitem{Построение вектора очереди правил на основе множества правил. Цикл по всем правилам и пока не достигнута цель;}
        \begin{scnindent}
            \begin{scnrelfromvector}{циклические операции}
                \scnfileitem{Получение посылки логического правила;}
                \scnfileitem{Определение типа посылки (связка конъюнкци, дизъюнкции, отрицания или атомарная логическая формула);}
                \scnfileitem{Проверка истинности посылки в зависимости от её типа;}
                \begin{scnindent}
                    \scntext{замечание}{Конъюнкция, дизъюнкция, отрицание работают нестабильно.}
                \end{scnindent}
                \scnfileitem{Генерация по шаблону следствия;}
                \scnfileitem{Добавление в дерево решений узла правила.}
                \begin{scnindent}
                    \scntext{примечание}{Смотрите пример выходной конструкции.}
                \end{scnindent}
            \end{scnrelfromvector}
        \end{scnindent}
        \scnfileitem{Возврат дерева применённых правил.}
    \end{scnrelfromvector}
\end{scnindent}

\begin{scnrelfromset}{недостатки текущего состояния}
    \scnfileitem{Импликация в текущем состоянии интерпретируется не как логическая связка, а как отношение \scnkeyword{выводимости*},
    то есть импликация не возвращает логическую константу, а генерирует новые знания. Вместо этого должны использоваться правила вывода,
    например, Modus ponens и другие правила, в процессе логического вывода.}
    \scnfileitem{В структуру ответа агента входит только узел solution, а не вся структура решения.}
    \scnfileitem{Нету реализации вызова агента в зависимости от переданных аргументов (старый или новый интерфейс).}
\end{scnrelfromset}

\begin{scnrelfromset}{преимущества текущего состояния}
    \scnfileitem{Проверка передаваемых параметров}
    \begin{scnindent}
        \scntext{примечание}{Если не передано множество правил, успешность завершения агента зависит от того, была ли достигнута цель.}
        \scntext{примечание}{Если не передана входная структура, используется База знаний IMS.}
        \scntext{примечание}{Если не передана выходная структура, результат генерации никуда не записывается.}
        \scntext{примечание}{Если не передан шаблон цели, агент проверяетвыполнимость каждого правила на входной структуре.}
    \end{scnindent}
    \scnfileitem{Сохранён интерфейс предыдущей реализации агента с тремя  аргументами.}
\end{scnrelfromset}
